\section{Software}

EL Google Chromecast no es más que un dispositivo compatible con el protocolo propietario de Google Cast que actúa como receptor.
Este protocolo fue lanzado en julio de 2013 en exclusividad para las aplicaciones de YouTube, Google Play Music, Google Play Movies \& TV y Netflix usando como receptor el Chromecast de primera generación, pero en febrero de 2014 pusieron el SDK a disposición de todos los desarrolladores para usarlo en sus propias aplicaciones.
En mayo de 2015 había más de 20.000 aplicaciones de terceros compatibles con esta tecnología según reconoce Google.

\

Para iniciar la reproducción de un contenido pulsamos el botón de \textit{cast}, que aparecerá automáticamente si Google Cast está integrado con la aplicación.
En ese momento aparecen los dispositivos Chromecast conectados a la red local y se elige aquel donde se quiere emitir el contenido.
Si el puerto HDMI dispone de Consumer Electronics Control (CEC) la televisión se encenderá inmediatamente.

\

Google Cast tiene dos modos de funcionamiento:

\begin{itemize}

	\item Uno es usar dispositivo desde el que solicitamos el streaming para controlar la reproducción (en adelante dispositivo emisor): pausar un vídeo, subir el volumen del audio, crear o modificar una cola de reproducción, etc.
	El dispositivo receptor (por ejemplo un Chromecast) es quien se encarga de descargarlo y comunicarse con el servidor de contenido, liberando al dispositivo emisor de esta tarea.
	Esto garantiza una carga de trabajo muy baja para el emisor y le permite estar bloqueado mientras la reproducción está teniendo lugar.
	El contenido puede estar almacenado localmente en el dispositivo emisor o en un servidor externo.
	El primer caso ocurre en aplicaciones como Google Photos, mientras que un ejemplo del segundo serían Netflix o YouTube.
	Las aplicaciones emisoras que usen este modo de funcionamiento deben ser compatible con Android 4.1, iOS 7.0 o versiones superiores si son aplicaciones móviles y con Windows 7, macOS 10.7, Chrome OS 28 o versiones superiores si son aplicaciones Chrome.
	En este último caso, se debe tener instalada la extensión Cast.

	\item El otro modo es hacer mirroring de una pestaña de Chrome o del escritorio de un ordenador con Chrome o del de un dispositivo con Android 4.4 o superior.
	La calidad del streaming en este caso varía ampliamente según la potencia de procesamiento del emisor.
	En el caso de hacerse desde un smartphone, la calidad de las imágenes normalmente se deteriora debido al escalado.

\end{itemize}

\begin{figure*}[h]
	\centering
	\begin{minipage}[b]{.35\textwidth}
		\centering
		\includegraphics[scale=0.65]{./Imagenes/ChromecastModo1.png}
		\caption{Ilustración del primer modo de funcionamiento de Google Cast}\label{fig:modo1}
	\end{minipage}\qquad
	\hspace{1cm}
	\begin{minipage}[b]{.35\textwidth}
		\centering
		\includegraphics[scale=0.65]{./Imagenes/ChromecastModo2.png}
		\caption{Ilustración del segundo modo de funcionamiento de Google Cast}\label{fig:modo2}
	\end{minipage}
\end{figure*}

Hasta diciembre de 2014, el dispositivo emisor y receptor debían estar conectados a la misma red Wi-Fi para reproducir contenido, pero en las versiones posteriores a esa fecha ya no es necesario.
Esto se debe a que se ha añadido un modo invitado.
En este modo, el receptor emite ultrasonidos a través de los altavoces y el emisor es capaz de localizarlo usando el micrófono.
Se usa un PIN de cuatro dígitos que aparece en pantalla para la verificación.
El modo invitado está disponible para todos los Chromecast con un dispositivo Android como emisor y para los Chromecast a partir de la segunda generación para aquellos con dispositivos iOS como emisor.

\

Como hemos adelantado, la API de Google Cast implementa el paradigma del productor-consumidor. Para implementar el protocolo hacen falta dos aplicaciones:

\begin{itemize}

	\item La aplicación emisora que se encarga de proveer al usuario la capacidad de controlar la reproducción y elegir el dispositivo donde se emite el contenido.
	Esta aplicación crea un canal seguro con la aplicación receptura para el intercambio de mensajes.

	\item La aplicación receptora es una web app ejecutándose en una versión adaptada del navegador Chrome con una interfaz gráfica en CSS.
	La aplicación receptora puede tener una complejidad muy variable, pudiendo ir desde limitarse a reproducir contenido HTML5 hasta soportar protocolos de streaming como MPEG-DASH, HTTP Live Streaming o el Microsoft Smooth Streaming Protocol\cite{DLNA-Miracast}.

\end{itemize}

Los formatos multimedia a los que Google Cast da soporte son los siguientes:

\begin{itemize}

	\item Imágenes en formato BMP, GIF, JPEG, PNG y WEBP, con un límite de 1280x720 píxeles de resolución.

	\item Los codecs de audio HE-AAC, LC-AAC, MP3, Vorbis, WAV (LPCM) y FLAC. AC-3 (Dolby Digital) y E-AC-3 (EC-3, Dolby Digital Plus) están disponibles para passthrough de audio.

	\item Los codecs de vídeo H.264 High Profile Level 4.1 (decodificación hasta 720/60 o 1080/30) y VP8.

\end{itemize}

En el CES de 2015, Google anunció una expansión de Google Cast centrada en la reproducción de audio.
La idea era que los fabricantes de altavoces integraran la tecnología Google Cast sin necesidad de depender de un Chromecast.
Está disponible en varios modelos de LG y Sony.

\

En mayo de 2015, Google lanzó nuevas APIs dirigidas a poder el televisor como segunda pantalla que muestre un contenido distinto del de la aplicación emisora.
Esto, junto con las Game Manager APIs, permite, por ejemplo, usar varios dispostivos como mandos en una partida de un videojuego y una pantalla común que proyecte la partida.
Uno de esos dispositivos sería el que controlara el estado de la partida y se sincronizarían entre ellos intercambiando mensajes con un Chromecast u otro dispositivo receptor.

\begin{figure*}[h]
	\centering
	\begin{minipage}[b]{.35\textwidth}
		\includegraphics[scale=0.3]{./Imagenes/games.png}
		\caption{Ilustración de Google para explicar el potencial de su Game Manager API}\label{fig:games}
	\end{minipage}\qquad
	\hspace{1cm}
	\begin{minipage}[b]{.35\textwidth}
		\includegraphics[scale=0.3]{./Imagenes/seconddisplay.png}
		\caption{Ilustración de Google como ejemplo de uso de una pantalla externa a través de Google Cast}\label{fig:seconddisplay}
	\end{minipage}
\end{figure*}

\begin{figure}[h]
	\centering
	\includegraphics[width=0.6\textwidth]{./Imagenes/gameexample.jpg}
	\label{fig:fondo}
	\caption{Ejemplo de uso de una pantalla externa para videojuegos}
\end{figure}


\subsection{Protocolos de detección de dispositivos}

La primeras versiones del Chromecast usaban el protocolo DIAL para la detección de dispositivos receptores desde la aplicación emisora.
En las últimas versiones, se utiliza el protocolo mDNS para cuando los dispositivos se encuentran en la misma red local y un protocolo especial basado en ultrasonidos para el modo invitado.

\subsubsection{DIAL}
DIAL (DIscovery And Launch) es el antiguo protocolo de comunicación que usaba Chromecast, desarrollado con Netflix y Youtube.
Se basa en Universal Plug and Play (UPnP), Simple Service Discovery Protocol (SSDP) y protocolos HTTP.
SSDP es un protocolo que sirve para la búsqueda de dispositivos UPnP en una red. Utiliza UDP en unicast o multicast en el puerto 1900 para anunciar los servicios de un dispositivo. Si el receptor ofrece el servicio deseado devuelve el mensaje '200 OK' con HTTP.


\begin{figure}[H]
	\centering
	\includegraphics[scale=0.5]{./Imagenes/dial.png}
	\caption{Funcionamiento DIAL}
	\label{fig:DIAL}
\end{figure}



DIAL permite a los llamados dispositivos en segundo plano, como tablets o móviles, enviar contenido a dispositivos en primer plano, por ejemplo televisiones.
Los dispositivos en segundo plano almacenarán el cliente, los de primer plano ejecutarán el servidor.
Para evitar conflictos con los nombres de aplicaciones DIAL tiene un registro con las aplicaciones que soporta. En el caso de que la aplicación deseada no esté en esa lista puede que tenga un prefijo que esté registrado, para así ofrecer mayor flexibilidad.

El protocolo dial tiene dos componentes, DIAL Service Discovery y DIAL REST Service.
El primero busca el dispositivo en primer plano dentro de una red local y permite conectarlos, trabajo que en la actualidad recae en mDNS.
El segundo permite al cliente mandar contenido, hacer consultas como subir o bajar volumen, etc. al servidor.
El servidor puede procesar mensajes de hasta 4KB.

\begin{figure}[H]
	\centering
	\includegraphics[scale=0.5]{./Imagenes/dialrest.png}
	\caption{Funcionamiento DIAL}
	\label{fig:DIAL}
\end{figure}


Las respuestas específicas y otros detalles como manejo de excepciones se pueden encontrar en \cite{dial}


\subsubsection{mDNS (multicast Domain Name System)}
Multicast Domain Name System (mDNS, \href{https://tools.ietf.org/html/rfc6763}{RFC 6763}) es el protocolo usado para encontrar los dispositivos a los que conectarnos.

mDNS transforma nombres de dominio en direcciones IP dentro de su registro de direcciones IPv4 o IPv6 usando el puerto 5353. Está pensado para redes locales sin tener que incluir un servidor DNS.
Tiene las mismas interfaces de programación, formatos y restricciones semánticas que DNS, pero permite designar una porción del espacio de nombres de DNS a nuestro gusto, sin tener que pagar anualmente.

Usa UDP en multicast y sus ventajas son:
\begin{itemize}
	\item Necesita poca configuración para activarse
	\item Funciona cuando no hay infraestructura
	\item Soporta fallos en la infraestructura
\end{itemize}

En el primer caso envía a todos los dispositivos una solicitud para identificarse.

\begin{figure}[H]
	\centering
	\includegraphics[width=0.6\textwidth]{./Imagenes/mdns1.png}
	\label{fig:mdns1}
	\caption{Primer paso del protocolo mDNS}
\end{figure}


Ejemplo de petición:

00 00 \textbf{00 00}(1) 00 01 00 00 \hspace{0.1cm} 00 00 00 00  \textbf{07 61 70 70}

\textbf{6c 65 74 76}(2) \textbf{05 6c 6f 63} \hspace{0.1cm} \textbf{61 6c}(3) 00 00 01 00 01
			
\begin{itemize}
	\item Flag de petición(1)
	\item Nombre de dominio del servidor (appletv(2).local(3))
\end{itemize}


Como respuesta se devuelve la dirección IP con una variable TTL (Time To Live) para saber cuánto tiempo durará vigente la IP enviada.

Si un dispositivo quiere rechazar la petición manda una IP con TTL cero.

\begin{figure}[H]
	\centering
	\includegraphics[width=0.6\textwidth]{./Imagenes/mdns2.png}
	\label{fig:mdns2}
	\caption{Segundo paso del protocolo mDNS}
\end{figure}


	Ejemplo respuesta:

	00 00 \textbf{84 00}(1) 00 00 00 01 \hspace{0.1cm} 00 00 00 02 07 61 70 70
	
	6c 65 74 76 05 6c 6f 63 \hspace{0.1cm} 61 6c 00 00 01 80 01 00
	
	00 78 00 00 04 \textbf{99 6d 07} \hspace{0.1cm} \textbf{5a}(2) c0 0c 00 1c 80 01 00
	
	00 78 00 00 10 \textbf{fe 80 00} \hspace{0.1cm} \textbf{00 00 00 00 00 02 23 32}
	
	\textbf{ff fe b1 21 52}(3) c0 0c 00 \hspace{0.1cm} 2f 80 01 00 00 78 00 00
	
	08 c0 0c 00 04 40 00 00 \hspace{0.1cm} 08

	
\begin{itemize}
	\item (1) Flag de respuesta
	\item (2) Bytes de dirección IPv4
	\item (3) Bytes de dirección IPv6
\end{itemize}


\subsubsection{Modo invitado}
Si se usa un dispositivo cercano a un Chromecast, este puede detectarlo y sincronizarse vía ultrasonidos.
El Chromecast emite sonidos a muy alta frecuencia (inaudibles para un humano) a través del altavoz de la televisión.
Estos sonidos son la codificación de un mensaje determinado.
A continuación, el emisor escucha este sondo a través del micrófono y lo transforma a un mensaje interpretable por la aplicación.
Este procedimiento está limitado por especificaciones del dispositivos: a menudo, el micrófono no es capaz de detectar sonidos a frecuencias tan altas, bien por limitaciones de hardware o de firmware.
Para entender mejor el funcionamiento del intercambio de información a través de ultrasonidos, se explica con más detalle en \href{http://smus.com/ultrasonic-networking/}{el blog del ingeniero de Google Boris Smus}.

\

Básicamente, hay que establecer un alfabeto de símbolos que se pueden usar en el mensaje, asociando a cada símbolo una frecuencia. Para traducción entre símbolos y sonidos se usa \href{https://en.wikipedia.org/wiki/Dual-tone_multi-frequency_signaling}{DTMF}.


\subsection{Comparativa con otros protocolos multimedia}

\subsubsection{Miracast}
Para que nuestra smartTV sea capaz de usar Miracast sin necesidad de Chromecast necesitaría soportar Wi-Fi Direct, es decir, estuviera conectada por Wi-Fi y fuese compatible con ella.

\begin{figure}[ht] 
	\begin{minipage}[b]{0.55\linewidth}
		La conexión está creada vía Wi-Fi Protected Setup (WPS), mecanismos para facilitar la configuración de una red WLAN con seguridad WPA2.
		WPS contempla cuatro configuraciones para el intercambio de credenciales, PIN (Personal Identification Number), PBC (Push Button Configuration), NFC (Near Field Communications) y USB (Universal Serial Bus). La configuración PIN no es recomendable por su debilidad ante ataques de fuerza bruta.
	\end{minipage}%%
	\begin{minipage}[b]{0.45\linewidth}
		\centering
		\includegraphics[width=.55\linewidth]{./Imagenes/miracast.jpg} 
	\end{minipage} 
\end{figure}



En la capa de internet es usado IPv4, en la capa de transporte es usado TCP o UDP. En la capa de aplicación el stream es inicializado y controlado por RTSP y RTP.
Los dispositivos que envían y reciben información tienen que estar certificados para Miracast, pero existe un plug para dispositivos no certificados.



\subsubsection{MiracleCast}
MiracleCast es una alternativa de código abierto a Miracast. El nombre, en palabras del autor, viene por que creía que necesitaba un milagro para crear una red Wifi-P2P estable (basado en $wpa_supplicant$), debido a los problemas que había tenido.

El núcleo de MiracleCast es un demonio llamado miracled \cite{MiracleCast}, que controla links locales, las peticiones de conexión, se encarga de
la codificación del protocolo y el parsing.
Su línea de comandos puede ser usada para controlar el demonio, crear nuevas conexiones, modificar parámetros, etc.
Soporta un modo interactivo que muestra las peticiones de conexión y permite al usuario aceptarlas o no.

El código fuente se puede encontrar en \href{https://github.com/albfan/miraclecast}{github}.

\subsection{Chrome OS?}
