\section{Software}
Usa el paradigma del productor consumidor, tiene una aplicación que envía y otra que recibe.
\begin{itemize}	
	\item La aplicación de envío ofrece controles para elegir el dispositivo de llegada.
	\item La aplicación receptora utiliza una aplicación web ejecutada en un entorno de Chrome del dispositivo. Se pueden hacer aplicaciones receptoras que aparte de soportar HTML5 tengan más variedad de protocolos de streaming: MPEG-DASH, HTTP Live Streaming, y Microsoft Smooth Streaming Protocol.
\end{itemize}

Chromecast para buscar dispositivos disponibles en una red Wi-Fi usa el protocolo mDNS (multicast Domain Name System), anteriormente usaba el protocolo DIAL (DIscovery And Launch).

Para la visualización de contenidos en la TV se mezclan conceptos DLNA (Digital Living Network Alliance) y Miracast\cite{DLNA-Miracast}.
A la hora de enviar contenido realmente se manda la orden a Chromecast de que reproduzca el contenido elegido directamente desde la nube.
Es la principal diferencia con DLNA, ya que no se reproduce contenido desde un servidor DLNA sino desde Internet.
Para usar solamente DLNA para streaming necesitaríamos aplicaciones de terceros, por ejemplo iMediaShare.
Otra funcionalidad es poder enviar los contenidos de una pestaña del navegador Chrome a la TV en lo que podríamos denominar una conexión Miracast pura y dura, punto a punto.




\begin{figure}[ht] 
	\begin{minipage}[b]{0.55\linewidth}
	Utiliza un sistema operativo de escritorio llamado Chrome OS, siendo el navegador Google Chrome su principal herramienta de uso.  
	Chrome OS se basa en el proyecto de código abierto Chromium OS,5 que, a diferencia de Chrome OS, se puede compilar a partir del código fuente descargado.
	\end{minipage}%%
	\begin{minipage}[b]{0.45\linewidth}
		\centering
		\includegraphics[width=.65\linewidth]{./Imagenes/googlecastbrowser.jpg} 
	\end{minipage} 
\end{figure}







\subsection{Google Cast}
\subsubsection{Google Home}

\subsection{mDNS (multicast Domain Name System)}


\subsection{Miracast}
Para que nuestra smartTV sea capaz de usar Miracast sin necesidad de Chromecast necesitaría soportar Wi-Fi Direct, es decir, estuviera conectada por Wi-Fi y fuese compatible con ella.

\begin{figure}[ht] 
	\begin{minipage}[b]{0.55\linewidth}
		La conexión está creada vía Wi-Fi Protected Setup (WPS), mecanismos para facilitar la configuración de una red WLAN con seguridad WPA2.
		WPS contempla cuatro configuraciones para el intercambio de credenciales, PIN (Personal Identification Number), PBC (Push Button Configuration), NFC (Near Field Communications) y USB (Universal Serial Bus). La configuración PIN no es recomendable por su debilidad ante ataques de fuerza bruta.
	\end{minipage}%%
	\begin{minipage}[b]{0.45\linewidth}
		\centering
		\includegraphics[width=.55\linewidth]{./Imagenes/miracast.jpg} 
	\end{minipage} 
\end{figure}



En la capa de internet es usado IPv4, en la capa de transporte es usado TCP o UDP. En la capa de aplicación el stream es inicializado y controlado por RTSP y RTP.
Los dispositivos que envían y reciben información tienen que estar certificados para Miracast, pero existe un plug para dispositivos no certificados.



\subsubsection{MiracleCast}
MiracleCast es una alternativa de código abierto a Miracast. El nombre, en palabras del autor, viene por que creía que necesitaba un milagro para crear una red Wifi-P2P estable (basado en $wpa_supplicant$), debido a los problemas que había tenido.

El núcleo de MiracleCast es un demonio llamado miracled \cite{MiracleCast}, que controla links locales, las peticiones de conexión, se encarga de
la codificación del protocolo y el parsing.
Su línea de comandos puede ser usada para controlar el demonio, crear nuevas conexiones, modificar parámetros, etc.
Soporta un modo interactivo que muestra las peticiones de conexión y permite al usuario aceptarlas o no.

El código fuente se puede encontrar en \href{https://github.com/albfan/miraclecast}{github}.


\subsection{Chrome OS?}