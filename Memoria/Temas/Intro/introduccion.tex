\section{Introducción}
Google Chromecast es un dispositivo de reproducción multimedia fabricado por Google y comercializado a partir de Julio de 2013. Reproduce contenido multimedia conectado a una televisión o monitor vía HDMI haciendo streaming mediante Wi-Fi.

\

Para hacer streaming utiliza el software Google Cast, un protocolo propietario de Google que permite controlar la reproducción multimedia de un receptor desde un dispositivo local. Google Cast dispone de librerías para las últimas versiones de Android y iOS, así como para Chrome OS y aplicaciones de Google Chrome.

\

Chromecast permite reproducir contenido almacenado en un dispositivo conectado a una red local o en un sevidor externo. El control de la reproducción se realiza en ambos casos desde uno o varios dispositivos locales compatibles con la tecnología Google Cast. Cuando no hay contenido en streaming reproduce un contenido personalizable de fondo, puede incluir fotos personales,
de satélite, noticias, etc. Por defecto muestra imágenes aleatorias seleccionadas por Google.

\

Para reproducción exclusiva de sonido existe un Chromecast Audio pensado para conectarse a altavoces. Hay otras tres versiones del Chromecast: la primera generación, la segunda y el Chromecast Ultra. La principal mejora de la segunda generación fue la compatibilidad con redes Wi-Fi a 5GHz. El Chromecast Ultra incluye como novedad la posibilidad de reproducir vídeo 4K.

\

Su principal competidor es el servicio AirPlay de Apple, que permite streaming inalámbrico entre dispositivos iPhone, iPad o Mac para audio, vídeo, fotos, etc.
