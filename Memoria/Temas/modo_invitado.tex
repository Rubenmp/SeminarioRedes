Si se usa un dispositivo cercano a un Chromecast, este puede detectarlo y sincronizarse vía ultrasonidos.
El Chromecast emite sonidos a muy alta frecuencia (inaudibles para un humano) a través del altavoz de la televisión.
Estos sonidos son la codificación de un mensaje determinado.
A continuación, el emisor escucha este sondo a través del micrófono y lo transforma a un mensaje interpretable por la aplicación.
Este procedimiento está limitado por especificaciones del dispositivo: a menudo, el micrófono no es capaz de detectar sonidos a frecuencias tan altas, bien por limitaciones de hardware o de firmware.
Para entender mejor el funcionamiento del intercambio de información a través de ultrasonidos, se explica con más detalle en \href{http://smus.com/ultrasonic-networking/}{el blog del ingeniero de Google Boris Smus}.
