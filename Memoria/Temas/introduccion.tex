\section{Introducción}
Google Chromecast es un dispositivo de reproducción multimedia fabricado por Google y comercializado a partir de Julio de 2013. Reproduce contenido multimedia conectado a una televisión o monitor vía HDMI haciendo streaming mediante Wi-Fi. Un nuevo modelo Chromecast Ultra que soporta 4k fue anunciado durante el evento \#MadeByGoogle.

\

Para hacer streaming utiliza el software Google Cast, que implementa distintos protocolos y tiene varios modos de funcionamiento que desarrollaremos a continuación.

\

Google Cast dispone de librerías para las últimas versiones de Android y iOS, así como para Chrome OS y aplicaciones de Google Chrome.

\begin{figure*}[h]
	\centering
	\begin{minipage}[b]{.35\textwidth}
		\includegraphics[scale=0.11]{./Imagenes/chromecast1gen.jpg}
		\caption{Primera generación}\label{fig:1gen}
	\end{minipage}\qquad
	\hspace{1cm}
	\begin{minipage}[b]{.35\textwidth}
		\includegraphics[scale=0.15]{./Imagenes/Chromecast.jpg}
		\caption{Segunda generación}\label{fig:2gen}
	\end{minipage}
\end{figure*}

Chromecast permite reproducir contenido almacenado en un dispositivo conectado a la red local o en un sevidor externo. El control de la reproducción se realiza en ambos casos desde uno o varios dispositivos locales compatibles con la tecnología Google Cast.

\

Para iniciar la reproducción pulsamos el botón de \textit{cast}. En ese momento aparecen los dispositivos Chromecast conectados a la red local y se elige aquel donde se quiere emitir el contenido. Si el puerto HDMI dispone de Consumer Electronics Control (CEC) la televisión se encenderá inmediatamente.

\

Cuando no hay contenido en streaming reproduce un contenido personalizable de fondo, puede incluir fotos personales,
de satélite, noticias, etc. Por defecto muestra imágenes aleatorias seleccionadas por Google.

\

Su principal competidor es el servicio AirPlay desarrollado por Apple, que permite streaming inalámbrico entre dispositivos iPhone, iPad o Mac para audio, vídeo, fotos, etc.

\vspace{1cm}
\begin{figure}[h]
	\centering
	\includegraphics[width=0.6\textwidth]{./Imagenes/fondo.png}
	\label{fig:fondo}
\end{figure}

\newpage


\subsection{Generaciones}

El chromecast de primera generación incluye un decodificador de VP8 y H.264 para formatos de compresión de vídeo, 512 MB de Micron DDR3L RAM y 2 GB de memoria flash.
El de segunda generación tiene un cable flexible y magnético, usa procesador dual ARM Cortex-A7 de frecuencia 1.2 GHz y tiene tres antenas adaptativas para mejroar la conexión con el router.
El dispositivo tiene 512 MB de Samsung DDR3L RAM y 256 MB de memoria flash.
