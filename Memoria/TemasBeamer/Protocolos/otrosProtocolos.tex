\subsection{Comparación otros protocolos}

\subsubsection{Mirascat}

\begin{frame}{Miracast}
	\begin{block}{ }
		Miracast es un protocolo multimedia para hacer streaming a un monitor desde un dispositivo local.

		Con Miracast el dispositivo receptor es dependiente de que el dispositivo Android emisor se mantenga activo: si se bloquea también bloqueará la reproducción en el receptor.
	\end{block}
	
	\begin{block}{Capas}
		\begin{itemize}
			\item Capa de internet: IPv4
			\item Capa de transporte: TCP/UDP
			\item Capa de aplicación, RTSP y RTP
		\end{itemize}
	\end{block}
\end{frame}

\begin{frame}
	\begin{block}{Red}
		La conexión está creada vía Wi-Fi Protected Setup
		(WPS), mecanismos para facilitar la configuración de
		una red WLAN con seguridad WPA2.
		
		Existe una alternativa de código abierto a Miracast llamada \href{https://github.com/albfan/miraclecast}{MiracleCast}.
	\end{block}
	
	\begin{alertblock}{Sin soporte de Google}
		A partir de Android 6.0, Google ha dejado de dar soporte nativo a Miracast en favor de su propio Google Cast.
	\end{alertblock}
\end{frame}
