\documentclass[11pt,spanish]{article} % Idioma
\usepackage{babel}
\usepackage[T1]{fontenc}
\usepackage{textcomp, verbatim} % \begin{comment}
\usepackage[utf8]{inputenc} % Permite acentos

\usepackage{wrapfig} % Imagenes %\graphicspath{ {./imagenes/} }
\usepackage[left=2.75cm,top=2.5cm,right=2cm,bottom=2.5cm]{geometry} % Márgenes
\usepackage{amssymb, amsmath, amscd, amsfonts, amsthm, mathrsfs } % Símbolos matemáticos
\usepackage{cancel} % Cancelar expresiones
\usepackage{multirow, multicol, tabularx, booktabs, longtable} % Tablas
\usepackage{fancyhdr, fncychap} % Encabezados
\usepackage{algpseudocode, algorithmicx, algorithm} % Pseudo-código	
\usepackage{bbding} % Símbolos
\usepackage{enumitem} % Enumerados a), b), c)... usando \begin{enumerate}[label=\alph*)]
\usepackage{graphicx, xcolor, color, pstricks} % Gráficos --TikZ-- 
% http://www.texample.net/tikz/examples/
\usepackage[hidelinks]{hyperref}  % Enlaces
\usepackage{verbatim} % Comentarios largos \begin{comment}
\usepackage{rotating} % \begin{rotate}{30}
\usepackage[all]{xy} % Diagramas
\usepackage{listings} % Escribir código 
\usepackage{xparse} % Entornos
\usepackage{hyperref}
\hypersetup{
	colorlinks=true,
	linkcolor=black,
	filecolor=magenta,      
	urlcolor=cyan,
}



% Comandos
\newcommand{\docdate}{}
\newcommand{\subject}{}
\newcommand{\docauthor}{Rubén Morales Pérez}
\newcommand{\docemail}{srmorales@correo.ugr.es}

\newcommand{\N}{\mathbb{N}}
\newcommand{\Q}{\mathbb{Q}}
\newcommand{\C}{\mathbb{C}}
\newcommand{\R}{\mathbb{R}}
\newcommand{\Z}{\mathbb{Z}}


\linespread{1.1}                  % Espacio entre líneas.
\setlength\parindent{0pt}         % Indentación para párrafo.

\title{Prácticas Fundamentos de Redes: \\ Chromecast}
\author{Francisco Javier Morales Piqueras \\
		María Florencio \\
		Rubén Morales Pérez}
\date{\today}

% % % % % % % % % % % % % % % % % % % % % % % % % % % % % % % % %
%					 Inicio del documento
% % % % % % % % % % % % % % % % % % % % % % % % % % % % % % % % %
\begin{document}

\maketitle
\tableofcontents % Generando el indice
\newpage



\section{Introducción}
Google Chromecast es un dispositivo de reproducción multimedia fabricado por Google y comercializado a partir de Julio de 2013. Reproduce audio/video conectado con una televisión o monitor por HDMI haciendo streaming mediante Wi-Fi. Un nuevo modelo Chromecast Ultra que soporta 4k fue anunciado durante el evento $\#$MadeByGoogle.

Para enviar la información utiliza Google Cast, seleccionamos el contenido que queremos reproducir por una aplicación (Android 4.1+, iOS 7.0+) o mediante el navegador chrome y se carga por su puerto mico-USB.

\begin{figure*}[h]
	\centering
	\begin{minipage}[b]{.35\textwidth}
		\includegraphics[scale=0.11]{./Imagenes/chromecast1gen.jpg}
		\caption{Primera generación}\label{fig:1gen}
	\end{minipage}\qquad
	\hspace{1cm}
	\begin{minipage}[b]{.35\textwidth}
		\includegraphics[scale=0.15]{./Imagenes/Chromecast.jpg}
		\caption{Segunda generación}\label{fig:2gen}
	\end{minipage}
\end{figure*}

Para iniciar la reproducción pulsamos el botón de \textit{cast}, si el puerto HDMI dispone de Consumer Electronics Control (CEC) la televisión se encenderá inmediatamente.
Dicho contenido puede residir también en su almacenamiento local. 
Con la función invitado pueden usarse redes Wi-Fi diferentes.

Las imágenes o vídeos enviados mediante dispositivos Android suelen perder calidad al estar escaladas las imágenes en pantallas pequeñas.
Cuando no hay contenido en streaming reproduce un contenido personalizable de fondo, puede incluir fotos personales, de satélite, noticias, etc.

Su principal competidor es el protocolo propietario AirPlay desarrollado por Apple Inc. que permite streaming inalámbrico entre dispositivos para audio, video, fotos, etc. 
\vspace{1cm}
\begin{figure}[h]
	\centering
	\includegraphics[width=0.6\textwidth]{./Imagenes/fondo.png}
	\label{fig:fondo}
\end{figure}

\newpage


\subsection{Generaciones}
El chromecast de primera generación incluye un decodificador de VP8 y H.264 para formatos de compresión de vídeo, 512 MB de Micron DDR3L RAM y 2 GB de memoria flash.
El de segunda generación tiene un cable flexible y magnético, usa procesador dual ARM Cortex-A7 de frecuencia 1.2 GHz y tiene tres antenas adaptativas para mejroar la conexión con el router.
El dispositivo tiene 512 MB de Samsung DDR3L RAM y 256 MB de memoria flash.
\section{Software}

\begin{frame}{Software}
	\begin{block}{Google Cast}
		Google Chromecast es un dispositivo que actúa como receptor y es compatible con el protocolo propietario Google Cast.
		Para iniciar la reproducción de un contenido pulsamos el botón de \textit{cast}.
	\end{block}

	\begin{figure}[h]
		\centering
		\includegraphics[width=0.7\textwidth]{./Imagenes/cast-speaker.jpg}
	\end{figure}
\end{frame}



\subsection{Modos de funcionamiento}

\begin{frame}{Funcionamiento}
	\begin{block}{Primer modo}
		Usar el dispositivo emisor para controlar la reproducción. El receptor (ej: Chromecast) se encarga de descargarlo del servidor, liberando al emisor de esta tarea. 
		Esto permite al emisor ahorrar batería, estar bloqueado o en otra aplicación mientras la reproducción tiene lugar.
	\end{block}
	
	\begin{block}{Segundo modo}
		Diseñado para enviar contenido del emisor, como cuando hacemos mirroring o usamos la televisión como segunda pantalla.
		
		La calidad del streaming en este caso varía según la potencia de procesamiento del emisor. En el caso de un smartphone la calidad
		de las imágenes normalmente se deteriora debido al escalado.
	\end{block}
\end{frame}


\begin{frame}{Comparativa}
	\begin{minipage}[b]{.45\textwidth}
		\centering
		\includegraphics[scale=0.52]{./Imagenes/ChromecastModo1.png}
		%\caption{Primer modo}\label{fig:modo1}
	\end{minipage}\qquad
	\hspace{1.65cm}
	\begin{minipage}[b]{.3\textwidth}
		\centering
		\includegraphics[scale=0.52]{./Imagenes/ChromecastModo2.png}
		%\caption{Segundo modo}
	\end{minipage}
\end{frame}


\begin{frame}{ }
	En versiones posteriores a 2014 no es necesario que emisor y receptor estén conectados a la misma red local al haber añadido un modo invitado.
	En este modo la comunicación tiene lugar mediante ultrasonidos, también puede usarse un PIN de cuatro dígitos que aparece en pantalla. 
\end{frame}



\subsection{Implementación}
\begin{frame}
	Google Cast implementa el paradigma del productor-consumidor.
	\begin{block}{ }
		La aplicación emisora se encarga de controlar la reproducción y elegir el dispositivo donde se emite el contenido. 
	\end{block}
	
	\begin{block}{ }
		La aplicación receptora es una aplicación web ejecutándose en una adaptación de Chrome.
		
		El código de la misma debe estar alojado en un servidor, ya que el Chromecast no almacena aplicaciones. Por tanto aunque el contenido esté alojado en un dispositivo de la red local, seguirá necesitando conexión a internet para cargar la web app. 
	\end{block}
\end{frame}




\section{Aplicaciones}
En el primer lanzamiento YouTube y Netflix eran soportadas como aplicaciones web en Android, iOS, y navegador Chrome, Google Play Music y Google Play Movies $\&$ TV eran soportadas como aplicaciones.
El SDK estuvo abierto para desarrolladores a partir de Febrero de 2014, ahora es parte del framework de Google Play Services.

Una lista completa de las aplicaciones compatibles se puede encontrar en la 
\href{https://www.google.com/intl/en/chromecast/apps/}{página web de chromecast}





%%%%%%%%%%%%%%%%%%%%%%%%%%%%%%%%%%%%%%%%%%%%%%%%%%%%%%%%%%%%%%%%%%%%%%%%%%%%%%%%%%

% % % % % % % % % % % % % % % % % % % % % % % % % % % % % % % % %
%					 Bibliografía
% % % % % % % % % % % % % % % % % % % % % % % % % % % % % % % % %

\end{document}
